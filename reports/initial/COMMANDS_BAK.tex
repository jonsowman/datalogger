\subsection{Data Acquisition}
    To begin datalogging, a configuration command (\texttt{CFG}) 
    is first sent to the analyser
    to set all configurable options, which would otherwise be a in default power
    up state. These options include sample rate and synchronous or asynchronous
    modes.

    An \texttt{ARM} command is then sent to the analyser, which will either wait
    for a change on channel 1 (clock) (in synchronous mode) or begin logging
    immediately (in asynchronous mode). The analyser will confirm beginning
    sampling by returning a START (\texttt{SRT}) packet to the PC, which 
    will then
    wait until an \texttt{END} command is received signalling the end of
    sampling. The PC will be able to cancel the logging by sending a CANCEL
    (\texttt{CNL}) command at any point during the logging process. During
    logging, the MCU will clock data into SRAM at the sample rate given, or on
    the rising/falling (configurable) edge of channel 1 (clock).

\subsection{Data Recovery}
    The PC will know when sampling has completed and can then request the sample
    data from the analyser. The MCU will confirm the start of data recovery, and
    then proceed to packetise the samples into 64 byte packets, which includes
    the \texttt{CMD} and \texttt{LEN} fields. These packets are transmitted over
    the USB interface until the SRAM is empty, with termination by a
    \texttt{END} command packet. Data is lost from the volatile SRAM on power
    removal, so data recovery must be completed before the unit is powered down.

